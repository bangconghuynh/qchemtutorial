\renewcommand*{\codelocation}{codesnippets/singlepoint}
\tikzsetexternalprefix{./singlepoint/tikz/}

\section{Single-Point Calculations}

		\subsection{Restricted Hartree--Fock (UHF)}
	
		We begin with a simple Hartree--Fock (HF) calculation for the simplest closed-shell system, a \ce{He} atom, to familiarise ourselves with the syntax of \textsc{Q-Chem}.
		Every \textsc{Q-Chem} run requires an input file in which the molecular configuration and all relevant calculation parameters are specified.
		\Cref{listing:He.RHF.STO-3G.inp} gives an example of such an input file.
		%
			\lstinputlisting[label={listing:He.RHF.STO-3G.inp}, language=QChemInput]{\codelocation/He.RHF.STO-3G.inp}
		
		We observe the following in \cref{listing:He.RHF.STO-3G.inp}:
			\begin{enumerate}
				\item There are two sections in this file: \qchemsection{molecule} specifies the molecular configuration of the system and \qchemsection{rem} specifies the parameters to be used in the calculation.
				\item The system of interest consists of a single \ce{He} nucleus located at the origin (line~\texttt{3}).
				\item The overall charge of the system is zero (first number on line~\texttt{2}), so there are two electrons in total.
				\item The ``spin multiplicity'' of the system is supposed to be \SI{1}{} (second number on line~\texttt{2}).
				However, one cannot in general constrain the value of $S$ in a conventional HF calculation.
				Therefore, the ``multiplicity'' specified here is actually equal to $2M_S + 1$ instead of $2S + 1$.
				This system thus has $M_S = 0$, but the existence and meaningfulness of its $S$ value is dependent upon the HF method and the solutions obtained.
				\item A self-consistent-field (SCF) restricted HF (RHF) calculation will be run on this system in the STO-3G basis set using the DIIS converging algorithm starting from the guess molecular orbital (MO) coefficients obtained by diagonalising the one-electron Hamiltonian matrix (lines \texttt{6}--\texttt{11}).
				\item The convergence criterion is when the maximum DIIS error becomes smaller than \SI{1e-13}{} (line~\texttt{12}). This is a very tight criterion.
				\item The maximum number of SCF cycles is \SI{1000}{} (line~\texttt{13}). This is more than enough here because convergence can be reached very quickly in this particular example.
				\item During each SCF cycle, the code maximises overlap with the orbitals from the previous cycle. This behaviour starts from cycle 1 (line~\texttt{14}) where the code maximises overlap with the guess orbitals.
				\item After each cycle, the code prints out only the minimal and useful output together with the component breakdown of SCF electronic energy (line~\texttt{15}).
				\item After convergence, the code prints out the occupied orbitals plus five virtual orbitals (line~\texttt{16}).
			\end{enumerate}
	
		To carry out the above \textsc{Q-Chem} calculation, copy \cref{listing:He.RHF.STO-3G.inp} from the \texttt{git} repository to a location of your choice and then execute the following command in a shell:
		%
			\begin{lstlisting}[style = custombash]
				qchem -nt 6 He.RHF.STO-3G.inp He.RHF.STO-3G.out
			\end{lstlisting}
		%
		which instructs \textsc{Q-Chem} to run a calculation with the input parameters specified in \texttt{He.RHF.STO-3G.inp} using six OpenMP threads and write the outputs to \texttt{He.RHF.STO-3G.out}.
		%
			\begin{Task}
				Examine the output file and determine the following:
				\begin{enumerate}[topsep=0pt,itemsep=-1ex,partopsep=1ex,parsep=1ex,label=(\alph*)]
					\item the DIIS error at convergence;
					\item the SCF energy of the converged solution;
					\item the contributing components of the SCF energy;
					\item the MO energies; and
					\item the MO coefficients in terms of the atomic orbital (AO) basis functions.
				\end{enumerate}
				What is the value of $\langle \hat{S}^2 \rangle$ of the converged solution and why? Assign a term symbol to this solution.
			\end{Task}

		
	\subsection{Unrestricted Hartree--Fock (UHF)}
		
		Duplicate \texttt{He.RHF.STO-3G.inp} and give it a different name such as \texttt{He.UHF.STO-3G.inp}.
		Then, change the value of \qchemremvar{UNRESTRICTED} on line~\texttt{9} to \texttt{true} as shown in \cref{listing:He.UHF.STO-3G.inp} and run \textsc{Q-Chem} on this new file as above to produce a new output file, say, \texttt{He.UHF.STO-3G.out}.
		%
			\lstinputlisting[label={listing:He.UHF.STO-3G.inp}, firstline=9, lastline=9, firstnumber=9, language=QChemInput]{\codelocation/He.UHF.STO-3G.inp}
		%
			\begin{Task}
				Examine \texttt{He.UHF.STO-3G.out} and compare the following with those in the RHF case: 
				\begin{enumerate}[topsep=0pt,itemsep=-1ex,partopsep=1ex,parsep=1ex,label=(\alph*)]
					\item the SCF energy of the converged solution;
					\item the contributing components of the SCF energy;
					\item the value of $\langle \hat{S}^2 \rangle$;
					\item the structure of the MO coefficients;
					\item the MO energies; and
					\item the MO coefficients in terms of the atomic orbital (AO) basis functions.
				\end{enumerate}
				Is this converged solution identical to that in the RHF case? If it is, explain the origin of the identicality.
			\end{Task}

		
	\subsection{Effects of the Basis Set}
	