% !TeX document-id = {355f5016-c7ca-4175-bb5e-82c4a2e94bb3}
% !TeX program = lualatex
% !TeX TXS-program:bibliography = txs:///biber

%% Preamble %%%%%%%%%%%%%%%%%%%%%%%%%%%%%%%%
% Class
\documentclass[%
	11pt,%
	oneside,%
	DIV=15,%
	BCOR=0mm,%
	headinclude=true,%
	footinclude=true,%
	parskip=half,%
	headsepline
]{scrartcl}


% Page layout
\usepackage{setspace}

\onehalfspacing

\setlength{\parindent}{1.5em}


% Maths support
\usepackage{amsmath, amssymb, mathrsfs, mathtools, gensymb, braket} %loading packages for further maths support

\usepackage{mleftright} %loading package for ensuring correct spacing before brackets
\mleftright


% Typography
\usepackage{csquotes}

\usepackage[usenames, svgnames]{xcolor} %loading package for colour support


% Diagrams and captions
\usepackage{pgfplots} % pgfplots loads tikz automatically

\usetikzlibrary{external}

\tikzexternalize

\newif\iftikzex
\tikzextrue
%\tikzexfalse

\iftikzex
\pgfplotsset{compat=1.16}

\usetikzlibrary{calc, luamath, positioning, pgfplots.groupplots, decorations.pathreplacing}

\pgfplotscreateplotcyclelist{coloronly}{%
	{red},%
	{blue},%
	{black!60!green},%
	{black!20!orange},%
	{green!30!brown},%
	{blue!40!red},%
	{black!60!blue},%
	{black!40!yellow},%
	{red!50!pink},%
	{green!70!blue},%
}
\fi

\makeatletter
\newcommand*{\useexternalfile}[1]{%
	\iftikzex
	\tikzsetnextfilename{tikzoutput/#1-output}%
	\scalebox{1}{\input{\tikzexternal@filenameprefix#1.tikz}}
	\else
	\includegraphics[scale=1]{\tikzexternal@filenameprefix tikzoutput/#1-output.pdf}
	\fi
}
\makeatother

\usepackage{xifthen} %conditionals in plots

\usepackage[%
format=plain,%
font={small,stretch=1.1},%
labelfont=bf,%
skip=5pt%
]{caption} %bold pre-texts in captions


% Chemistry
\usepackage[version=4]{mhchem} %loading package for further chemistry typesetting support


% Scientific typesetting
\usepackage{siunitx, graphicx}


%% Lists
\usepackage{enumitem}


% Micro-typography
\usepackage[%
	activate={true, nocompatibility},%
	final,%
	tracking=true,%
	factor=1100,%
	stretch=10,%
	shrink=10%
]{microtype}

\SetTracking{encoding={*}, shape=sc}{40} % Reduce spacing between sc characters

\microtypecontext{spacing=nonfrench}


% Font & languages
\usepackage[no-math]{fontspec}
\defaultfontfeatures{%
	Ligatures = TeX,%
	Numbers = Lining}

\usepackage{polyglossia} %loading package for language support; polyglossia requires fontspec, so load fontspec first
\setmainlanguage[variant=uk]{english}

\setmainfont{Alegreya}[%
	Ligatures	= TeX
]
\setsansfont{Alegreya-Sans}

\newcommand*\bmmax{2}
\usepackage{bm}

\DeclareMathSizes{8}{8}{5.5}{5}
\addtokomafont{disposition}{\rmfamily} %change all heading fonts to roman
\KOMAoptions{DIV=last}


\usepackage[math-style=TeX]{unicode-math}
\setmathfont{STIX2Math.otf}
\setmathfont{Alegreya-Italic.otf}[Scale=MatchLowercase, ScaleAgain=0.99999, range = it/{latin, Latin, greek, Greek}]
\setmathfont{Alegreya-Regular.otf}[Scale=MatchLowercase, ScaleAgain=0.99999, range = up/{latin, Latin, greek, Greek, num}]
\setmathfont{Alegreya-BoldItalic.otf}[Scale=MatchLowercase,ScaleAgain=0.99999,range=bfit/{latin, Latin, greek, Greek}]
\setmathfont{Alegreya-Bold.otf}[Scale=MatchLowercase,ScaleAgain=0.99999,range=bfup/{latin, Latin, greek, Greek, num}]


% Bookmarking
\usepackage[hypertexnames=false,hidelinks,hyperfootnotes=false]{hyperref} %loading package for hyperlinks in references and for creating pdf bookmarks automatically; hyperref must be loaded last
\hypersetup{bookmarksnumbered=true,colorlinks=true,citecolor=blue!60!black,linkcolor=blue!60!black} % this is for screen version
%\hypersetup{bookmarksnumbered=true,colorlinks=false} % this is for print version




% Bibliography
\usepackage[%
	sorting=none,%
	style=numeric-comp,%
	autocite=superscript,%
	dateabbrev=false,%
	url=false,%
	isbn=false,%
	backend=biber%
]{biblatex}
\addbibresource{bib/qchemtutorial.bib}


% Footnotes
\usepackage[perpage]{footmisc}


% TOC
\usepackage[toc]{multitoc}
\renewcommand*{\multicolumntoc}{2}
\setlength{\columnseprule}{0.5pt}


% Framed environment
\usepackage[framemethod = tikz]{mdframed}

\newcounter{Task}[section]
\renewcommand{\theTask}{\thesection.\arabic{Task}}
\newenvironment{Task}[1][]{%
	\refstepcounter{Task}%
	\tikzexternaldisable
	\ifstrempty{#1}%
	{\mdfsetup{%
			frametitle={%
				\tikz[baseline=(current bounding box.east),outer sep=0pt]
				\node[line width=1pt,anchor=east,rectangle,rounded corners=1ex,draw=black,fill=LightSalmon]
				{\strut \textbf{Task}~\theTask};}}
	}%
	{\mdfsetup{%
			frametitle={%
				\tikz[baseline=(current bounding box.east),outer sep=0pt]
				\node[line width=1pt,anchor=east,rectangle,rounded corners=1ex,draw=black,fill=LightSalmon]
				{\strut \textbf{Task}~\theTask:#1};}}%
	}%
	\tikzexternaldisable
	\mdfsetup{innertopmargin=10pt,linecolor=black,backgroundcolor=LightGoldenrodYellow,%
		linewidth=1pt,topline=true,%
		nobreak=true,
		frametitleaboveskip=\dimexpr-\ht\strutbox\relax,}
	\begin{mdframed}[]\relax%
	}{\end{mdframed}
}


% Code listing
\usepackage{listings, lstautogobble, accsupp}

% !TeX root = qchemtutorial.tex

\definecolor{eclipseBlue}{RGB}{42,0.0,255}
\definecolor{eclipseGreen}{RGB}{63,127,95}
\definecolor{eclipsePurple}{RGB}{127,0,85}

\lstdefinelanguage{QChem}
{
	%% List of keywords
	alsoletter = {_\$},
	% Rem variables
	keywords = [1]{
		SCF_MINFIND_INITNORM,
		BASIS,
		EXCHANGE,
		CORRELATION,
		UNRESTRICTED,
		SCF_GUESS,
		SCF_CONVERGENCE,
		SCF_ALGORITHM,
		SCF_MAX_CYCLES,
		MOM_START,
		PRINT_ORBITALS,
		SCF_PRINT,
	},
	keywords = [2]{
		\$molecule,
		\$rem,
		\$end,
	},
	comment = [l]{!},
	% Style
	sensitive=false, % keywords are not case-sensitive
	caption = \lstname,
	backgroundcolor = \color{Cornsilk},
	frame = trbl,
	% Syntax highlighting
	keywordstyle = [1]{\color{eclipsePurple}},
	keywordstyle = [2]{\color{Navy}},
	commentstyle = \color{eclipseGreen}
}

\lstdefinestyle{custombash}{
	language = bash,
	frame = trbl,
	backgroundcolor = \color{Seashell},
}

\newcommand*{\noncopy}[1]{
	\BeginAccSupp{method=escape, ActualText={}}
	#1
	\EndAccSupp{}
}

\newcommand*{\codelocation}

\lstset{
	basicstyle = \small\ttfamily, % Global Code Style
	extendedchars = true, % Allows 256 instead of 128 ASCII characters
	tabsize = 4, % number of spaces indented when discovering a tab
	columns = fixed, % make all characters equal width
	autogobble = true,
}

% Prepend a prompt to each bash line
% https://tex.stackexchange.com/questions/220790/how-can-i-automatically-prepend-a-prompt-e-g-to-each-line-of-a-listing
\makeatletter
% define custom macro that expands to the language name (for comparison purposes)
\newcommand*\langname@bash{}
\def\langname@bash{bash}

% define custom prompt
\newcommand*\prompt@bash{{\color{Green} \noncopy{\$}}}

% define a macro (initially empty) and insert it at the beginning of every paragraph
\newcommand*\addedToEveryPar@bash{}
\lst@AddToHook{EveryPar}{\addedToEveryPar@bash}

% redefine the macro by the custom prompt, but only if the language in use be `bash'
\lst@AddToHook{PreInit}{%
	\ifx\lst@language\langname@bash%
	\let\addedToEveryPar@bash\prompt@bash%
	\fi
}
\makeatother


% Cross-referencing
\usepackage[capitalise, noabbrev]{cleveref} %loading package for enhanced cross-referencing; cleveref must be loaded after hyperref and listings

\Crefname{Task}{Task}{Tasks}

%% Main document %%%%%%%%%%%%%%%%%%%%%%%%%%%
\begin{document}
	\vspace{1cm}
	\noindent
	\bgroup
	\renewcommand\arraystretch{1.5} \begin{tabular*}{\linewidth}{>{\centering\arraybackslash}m{\linewidth}}
		\hline
		\textbf{\Large Introduction to \textsc{Q-Chem} and \textsc{QCMagic}}\\
		\textit{\large A brief tutorial}\\
		\textcopyright\, Thom Group Cambridge, 2019\\[-8pt]
		Maintainer: Bang C. Huynh (\href{mailto:cbh31@cam.ac.uk}{\texttt{cbh31[at]cam.ac.uk}})\\
		\hline
	\end{tabular*}
	\egroup
	
	\setcounter{tocdepth}{2}
	\tableofcontents
	
	
	Before we begin, make sure that you have a copy of \textsc{Q-Chem} 5.2 compiled with parallelisation on your workstation and that you are sufficiently familiar with Unix shell commands to navigate around and perform various day-to-day tasks.
	This tutorial assumes that you already know how to carry out such tasks and focuses entirely on \textsc{Q-Chem} and \textsc{QCMagic}.
	
	\clearpage
	
	\renewcommand*{\codelocation}{codesnippets/singlepoint}
\tikzsetexternalprefix{./singlepoint/tikz/}

\section{Single-Point Calculations}

		\subsection{Restricted Hartree--Fock (RHF)}
	
		We begin with a simple Hartree--Fock (HF) calculation for the simplest closed-shell system, a \ce{He} atom, to familiarise ourselves with the syntax of \textsc{Q-Chem}.
		Every \textsc{Q-Chem} run requires an input file in which the molecular configuration and all relevant calculation parameters are specified.
		\Cref{listing:He.RHF.STO-3G.inp} gives an example of such an input file.
		%
			\lstinputlisting[label={listing:He.RHF.STO-3G.inp}, language=QChemInput]{\codelocation/He.RHF.STO-3G.inp}
		
		We observe the following in \cref{listing:He.RHF.STO-3G.inp}:
			\begin{enumerate}
				\item There are two sections in this file: \qchemsection{molecule} specifies the molecular configuration of the system and \qchemsection{rem} specifies the parameters to be used in the calculation.
				\item The system of interest consists of a single \ce{He} nucleus located at the origin (line~\texttt{3}).
				\item The overall charge of the system is zero (first number on line~\texttt{2}), so there are two electrons in total.
				\item The ``spin multiplicity'' of the system is supposed to be \SI{1}{} (second number on line~\texttt{2}).
				However, one cannot in general constrain the value of $S$ in a conventional HF calculation.
				Therefore, the ``multiplicity'' specified here is actually equal to $2M_S + 1$ instead of $2S + 1$.
				This system thus has $M_S = 0$, but whether or not $S$ is a ``good'' quantum number for the system is dependent upon the HF method and the solutions obtained.
				\item A self-consistent-field (SCF) restricted HF (RHF) calculation will be run on this system in the STO-3G basis set using the DIIS converging algorithm starting from the guess molecular orbital (MO) coefficients obtained by diagonalising the one-electron Hamiltonian matrix (lines \texttt{6}--\texttt{11}).
				\item The convergence criterion is when the maximum DIIS error becomes smaller than \SI{1e-13}{} (line~\texttt{12}). This is a very tight criterion.
				\item The maximum number of SCF cycles is \SI{1000}{} (line~\texttt{13}). This is more than enough here because convergence can be reached very quickly in this particular example.
				\item During each SCF cycle, the code maximises overlap with the orbitals from the previous cycle. This behaviour starts from cycle 1 (line~\texttt{14}) where the code maximises overlap with the guess orbitals.
				\item After each cycle, the code prints out only the minimal and useful output together with the component breakdown of SCF electronic energy (line~\texttt{15}).
				\item After convergence, the code prints out the occupied orbitals plus five virtual orbitals (line~\texttt{16}).
			\end{enumerate}
	
		To carry out the above \textsc{Q-Chem} calculation, copy \cref{listing:He.RHF.STO-3G.inp} from the \texttt{git} repository to a location of your choice and then execute the following command in a shell:
		%
			\begin{lstlisting}[style = custombash]
				qchem -nt 6 He.RHF.STO-3G.inp He.RHF.STO-3G.out
			\end{lstlisting}
		%
		which instructs \textsc{Q-Chem} to run a calculation with the input parameters specified in \texttt{He.RHF.STO-3G.inp} using six OpenMP threads and write the outputs to \texttt{He.RHF.STO-3G.out}.
		%
			\begin{Task}
				Examine the output file and determine the following:
				\begin{enumerate}[topsep=0pt,itemsep=-1ex,partopsep=1ex,parsep=1ex,label=(\alph*)]
					\item the DIIS error at convergence;
					\item the SCF energy of the converged solution;
					\item the contributing components of the SCF energy;
					\item the MO energies; and
					\item the MO coefficients in terms of the atomic orbital (AO) basis functions.
				\end{enumerate}
				What is the value of $\langle \hat{S}^2 \rangle$ of the converged solution and why? Assign a term symbol to this solution.
			\end{Task}

		
	\subsection{Unrestricted Hartree--Fock (UHF)}
		
		Duplicate \texttt{He.RHF.STO-3G.inp} and give it a different name such as \texttt{He.UHF.STO-3G.inp}.
		Then, change the value of \qchemremvar{UNRESTRICTED} on line~\texttt{9} to \texttt{true} as shown in \cref{listing:He.UHF.STO-3G.inp} and run \textsc{Q-Chem} on this new file as above to produce a new output file, say, \texttt{He.UHF.STO-3G.out}.
		%
			\lstinputlisting[label={listing:He.UHF.STO-3G.inp}, firstline=9, lastline=9, firstnumber=9, language=QChemInput]{\codelocation/He.UHF.STO-3G.inp}
		%
			\begin{Task}
				Examine \texttt{He.UHF.STO-3G.out} and compare the following with those in the RHF case: 
				\begin{enumerate}[topsep=0pt,itemsep=-1ex,partopsep=1ex,parsep=1ex,label=(\alph*)]
					\item the SCF energy of the converged solution;
					\item the contributing components of the SCF energy;
					\item the value of $\langle \hat{S}^2 \rangle$;
					\item the structure of the MO coefficients;
					\item the MO energies; and
					\item the MO coefficients in terms of the atomic orbital (AO) basis functions.
				\end{enumerate}
				Is this converged solution identical to the RHF one? If so, explain the origin of the identicality.
			\end{Task}

		
	\subsection{Effects of Basis Sets}
		Choosing an appropriate basis set is an important aspect of electronic structure calculation: a suitable basis set not only gives sensible and sufficiently accurate results but also saves computation time and storage space.
		To investigate basis set effects on the solutions obtained so far, duplicate \texttt{He.RHF.STO-3G.inp} and \texttt{He.UHF.STO-3G.inp} to \texttt{He.RHF.6-31GSTAR.inp} and \texttt{He.UHF.6-31GSTAR.inp}, respectively, then change the value of \qchemremvar{BASIS} on line~\texttt{6} to \texttt{6-31G*} as shown in \cref{listing:He.RHF.6-31GSTAR.inp,listing:He.UHF.6-31GSTAR.inp}.
 		Run \textsc{Q-Chem} on these files as above to produce \texttt{He.RHF.6-31GSTAR.out} and \texttt{He.UHF.6-31GSTAR.out}.
		%
			\lstinputlisting[label={listing:He.RHF.6-31GSTAR.inp}, firstline=6, lastline=6, firstnumber=6, language=QChemInput]{\codelocation/He.RHF.6-31GSTAR.inp}
		%
		%
			\lstinputlisting[label={listing:He.UHF.6-31GSTAR.inp}, firstline=6, lastline=6, firstnumber=6, language=QChemInput]{\codelocation/He.UHF.6-31GSTAR.inp}

			\begin{Task}
				Examine \texttt{He.RHF.6-31GSTAR.out} and \texttt{He.UHF.6-31GSTAR.out} and answer the following: 
				\begin{enumerate}[topsep=0pt,itemsep=-1ex,partopsep=1ex,parsep=1ex,label=(\alph*)]
					\item Are the RHF and UHF solutions located in 6-31G* identical to each other?
					\item Can these solutions be assigned to the same term as those in STO-3G? If so, are they a better description of this term and why?
				\end{enumerate}
			\end{Task}
		
		But how exactly does 6-31G* differ from STO-3G for a \ce{He} atom?
		One possible way to determine this is to ask \textsc{Q-Chem} to print out the basis set definition.
		This is achieved by setting \qchemremvar{PRINT\_GENERAL\_BASIS} to \texttt{true} in the \qchemsection{rem} section.
		Another way is to look it up on \href{https://www.basissetexchange.org/}{\texttt{https://www.basissetexchange.org/}}.
		
		Shown in \cref{listing:He.RHF.6-31GSTAR.withbasis.out} is the 6-31G* basis set definition for a \ce{He} atom as printed by \textsc{Q-Chem}.
		This tells us that, in 6-31G*, each electron in a \ce{He} atom is described by two $s$ shells. 
		The radial part of the first shell is constructed by contracting three primitive Gaussian functions whereas the radial part of the second shell is constructed by contracting only a single primitive Gaussian function.
		The Gaussian primitive exponents and the contraction coefficients are given by the first and second columns respectively within each shell.
		Since both of these shells are of angular momentum $s$, their angular parts are simply $1$.
		%
			\lstinputlisting[label={listing:He.RHF.6-31GSTAR.withbasis.out}, firstline=122, lastline=134, firstnumber=122, language=QChemOutput]{\codelocation/He.RHF.6-31GSTAR.withbasis.out}
		%

			\begin{Task}
				\begin{enumerate}[topsep=0pt,itemsep=-1ex,partopsep=1ex,parsep=1ex,label=(\alph*)]
					\item In 6-31G*, how many basis functions are used to describe each orbital of \ce{He}?
					\item How does STO-3G differ from 6-31G* for \ce{He}?
				\end{enumerate}
				
				6-31G** is another small basis set for \ce{He} in which each electron is described by two $s$ shells and one $p$ shell. Run an RHF calculation in this basis and answer the following:
					\begin{enumerate}[topsep=0pt,itemsep=-1ex,partopsep=1ex,parsep=1ex,label=(\alph*)]
							\setcounter{enumi}{2}
							\item How many basis functions are used to describe each orbital of \ce{He} in this basis set? Why is this number different from the number of shells in this basis set?
							\item Examine the orbital coefficient matrix carefully. Why does it have a very particular block-diagonal form?
							\item Does 6-31G** offer any improvement over 6-31G* for the ground RHF solution of \ce{He}?
					\end{enumerate}
			\end{Task}
	
	\clearpage
	\printbibliography[title={References}]
\end{document} 
